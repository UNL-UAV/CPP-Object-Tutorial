\documentclass[10pt]{exam}
\printanswers
\usepackage{dirtree}
\usepackage{epsfig,amssymb}
\usepackage{xcolor}
\definecolor{darkred}{rgb}{0.5,0,0}
\definecolor{darkgreen}{rgb}{0,0.5,0}
\usepackage{hyperref}
\usepackage{fullpage}
\usepackage{tikz}
\pagestyle{empty}
\usepackage{listings}
\setlength{\parindent}{0pt} %
\setlength{\parskip}{.25cm}
\newcommand{\comment}[1]{}
\boxedpoints
\addpoints
\usepackage{amsmath}
\usepackage{algorithm2e}
\usepackage{url}
\usepackage{enumitem}
\usepackage{graphics}
\usepackage{listings}
\pagestyle{headandfoot}
\firstpageheader{\Large UNL-UAV}{{\Large C++ Assignement}}{\Large Spring 2020}
\begin{document}
Before we begin we need to set up our workspace structure. Make sure you have premake5 installed.

Find decent directory to start your workspace in. I recommend starting a folder in \lstinline{~/git/} for all your git projects. Then when making a workspace just start a new folder in \lstinline{~/git/}. Make a new folder call \lstinline{cpp-tutorial}. Then make your file structure like the following:

\dirtree{%
	.1 cpp-tutorial.
	.2 include.
	.2 src.
	.2 premake5.lua.
}

In the premake5.lua file copy and paste the following:

\begin{lstlisting}[language={[5.0]lua},basicstyle=\ttfamily\footnotesize,keywordstyle=\bfseries]
outputdir = "%{cfg.buildcfg}/%{cfg.system}-%{cfg.architecture}"
workspace "CPP-Tutorial"
	startproject "Source"
	location "workspace"
	architecture "x64"

	configurations{
		"Debug",
		"Release",
		"Dist"
	}

	filter "configurations:Dist"
		postbuildcommands{
			"{COPY} bin/"..outputdir.."/* builds"
		}

	filter "system:windows"
		defines "_WINDOWS"

	filter "system:linux"
		defines "_LINUX" 

	filter "system:macosx"
		defines "_OSX"

	filter "configurations:Debug"
		defines "_DEBUG"
		symbols "On"
	filter "configurations:Dist"
		defines "_DIST"
		optimize "On"
	filter "configurations:Release"
		defines "_RELEASE"
		optimize "On"
project "Source"
	cppdialect "C++17"
	location "workspace/source"
	kind "ConsoleApp"
	language "C++"

	targetdir ("bin/" .. outputdir)
	objdir ("bin-int/" .. outputdir)

	files{
		"src/**.h",
		"src/**.hpp",
		"src/**.c",
		"src/**.cpp"
	}
	includedirs{
		"src",
		"include"
	}
\end{lstlisting}

Save the file the execute the following command in the root of the workspace.

\begin{lstlisting}
$ premake5 gmake2
\end{lstlisting}

Now you should see a workspace folder, here you is where you can compile your code using the following command. Your executable will be in \lstinline{bin/}.
\begin{lstlisting}
$ make
\end{lstlisting}

\newpage

\textbf{Notice:} For all objects make a \lstinline{.h} file in the include directory and make a corresponding \lstinline{.cpp}
\begin{questions}
\question Make UAV::Test::Printable with a pure virtual method called printable;
\question Make UAV::Test::Address that implments UAV::Test::Printable with standard postal fields with getter and setters.
\question Make a UAV::Test::Banking::Bank object that have the following fields and the propper getters.
\begin{itemize}
\item std::string name;
\item Address address;
\item std::vector[SafetyDepositBox] boxes;
\item destructor
\end{itemize}

\question Make a UAV::Test::Banking::SafteyDepositBox object with the following methods and fields:
\begin{itemize}
\item int boxNumber;
\item UAV::Test::Person person;
\end{itemize}

\question Make a UAV::Test::Person object that implements UAV::Test::Printable and have the following fields:
\begin{itemize}
\item std::string name;
\item int age;
\end{itemize}
\end{questions}
\end{document}