\documentclass[10pt]{exam}
\printanswers
\usepackage{dirtree}
\usepackage{epsfig,amssymb}
\usepackage{xcolor}
\definecolor{darkred}{rgb}{0.5,0,0}
\definecolor{darkgreen}{rgb}{0,0.5,0}
\usepackage{hyperref}
\usepackage{fullpage}
\usepackage{tikz}
\pagestyle{empty}
\usepackage{listings}
\setlength{\parindent}{0pt} %
\setlength{\parskip}{.25cm}
\newcommand{\comment}[1]{}
\boxedpoints
\addpoints
\usepackage{amsmath}
\usepackage{algorithm2e}
\usepackage{url}
\usepackage{enumitem}
\usepackage{graphics}
\usepackage{listings}
\pagestyle{headandfoot}
\firstpageheader{\Large UNL-UAV}{{\Large C++ Assignement}}{\Large Spring 2020}
\begin{document}
{\LARGE Explaination}

Before we get to our activity today we are go though unit testing.

Unit testing is a level of software testing where individual units/ components of a software are tested. 

What we can do with unit testing we can make sure a features is working and outputting the correct information, instead of going into gdb or make printf statements everywhere.

{\Large How do we use Unit Testing}

We are going to be using a libarary called catch2. Which provides an all in one header implmentation of unit testing.

\newpage
{\LARGE Activity}

Before we begin we need to set up our workspace structure. Make sure you have premake5 installed.

Find decent directory to start your workspace in. I recommend starting a folder in \lstinline{~/git/} for all your git projects. Then when making a workspace just start a new folder in \lstinline{~/git/}. Make a new folder call \lstinline{cpp-tutorial}. Then make your file structure like the following:

\dirtree{%
	.1 cpp-tutorial/.
	.2 include/.
	.2 src/.
	.2 test/.
	.2 premake5.lua.
}

Execute the following command:

\begin{lstlisting}
$ git submodule add https://github.com/catchorg/Catch2 vendor/catch2
\end{lstlisting}
What this do is pull in a clone of Catch2 as a dependancy. Think of this as a Git repo that links back to the orignal in our Git repo.

In the premake5.lua file copy and paste the code from the following website: https://git.root3287.site/snippets/2

We want to edit a few components.
\begin{enumerate}
\item Add the following under line 3
\begin{lstlisting}
includeDir["catch2"] = "vendor/catch2/single_include"
\end{lstlisting}
\item Copy Line 38-56 and paste it under line 56.
\item Change the string on the newly created project to "Test"
\item On the newly created project change the location varible to "workspace/test"
\item On the newly created project change all the src/ to test/
\end{enumerate}

Save the file the execute the following command in the root of the workspace.

If you have Gmake installed:
\begin{lstlisting}
$ premake5 gmake2
\end{lstlisting}

If you have visual studio installed:
\begin{lstlisting}
$ premake5 vs2019
\end{lstlisting}

If you have xcode installed:
\begin{lstlisting}
$ premake5 xcode4
\end{lstlisting}

Now you should see a workspace folder, here you is where you can compile your code using the following command. Your executable will be in \lstinline{bin/}.
Gmake:
\begin{lstlisting}
$ make
\end{lstlisting}

Other:
Open up the file with your file explorer.

\newpage

\textbf{Notice:} For all objects make a \lstinline{.h} file in the include directory and make a corresponding \lstinline{.cpp}
\begin{questions}
\question Make UAV::Test::Printable with a pure virtual method called printable;
\question Make UAV::Test::Address that implments UAV::Test::Printable with standard postal fields with getter and setters.
\question Make a UAV::Test::Banking::Bank object that have the following fields and the propper getters.
\begin{itemize}
\item std::string name;
\item Address address;
\item std::vector[SafetyDepositBox] boxes;
\item destructor
\end{itemize}

\question Make a UAV::Test::Banking::SafteyDepositBox object with the following methods and fields:
\begin{itemize}
\item int boxNumber;
\item float money;
\item UAV::Test::Person person;
\end{itemize}

\question Make a UAV::Test::Person object that implements UAV::Test::Printable and have the following fields:
\begin{itemize}
\item std::string name;
\item int age;
\end{itemize}

\question Make a test.cpp in the \lstinline{test/} folder. define CATCH CONFIG MAIN, and include catch2 main header. \textbf{Note}: You can just use standard assert

\question Create a test case for address using Catch2.
\question Create a test case for person using Catch2.
\question Create a test case for Safty deposit box using Catch2;
\end{questions}
\end{document}

